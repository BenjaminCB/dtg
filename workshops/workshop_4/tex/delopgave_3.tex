Umiddelbart er vi ikke kommet nærmere hvilke $x$, der opfylder \eqref{eq:Betingelse}. Vi kan dog vise, at $Q(x)$ og $R(x)$ egentlig er betingelser, der minder meget om hinanden.


\section{Task 3}
\subsection{Subtask 1}
\noindent
\textbf{Bevis følgende sætning: Lad $x$ være et positivt heltal. At $\gcd(x,2)=1$ er ækvivalent med, at $x$ er ulige.}

\bigskip
\noindent
\textbf{Solution:} We assume that $x=2n$ and that $gcd(x,2)=1$. If the previous two are true it means that 2 does not divide x. However, x is even meaning that we can write is $2n$ clearly seing that it is divisible by 2. We now have a contradiction, thus if $gcd(x,2)=1$ x has to be odd.

\subsection{Subtask 2}
\noindent
\textbf{Forklar hvorfor, at ovenstående betyder, at $Q(x)$ reelt set er det samme som $R(x)$. Derfor kan vi erstatte $R(x)$ med $Q(x)$ i \eqref{eq:Betingelse} og opnå, at \eqref{eq:Betingelse} er ækvivalent med
	\begin{align}\label{eq:BetingelseReduceret}
		&(P(x)\wedge \neg Q(x)) \vee \neg(P(x)\vee \neg Q(x) \vee Q(x)) \vee (\neg P(x) \wedge \neg Q(x) \wedge Q(x)) \nonumber\\ 
		\equiv& (P(x)\wedge \neg Q(x)) \vee \neg(P(x)\vee \mathbb{T}) \vee (\neg P(x) \wedge \mathbb{F}) \nonumber\\
		\equiv& P(x)\wedge \neg Q(x)
	\end{align}
}

\bigskip
\noindent
\textbf{Solution:} This means that if Q(x) is true x is odd on if Q(x) is false x is even. If we say that $9^2=n$ we get that $n mod 5 = 4$ for Q(x) be true. Now i just have to show that $9^x = 5m+4$ when x is odd and that it doesn't when it's even. I start with the base cases of $x=1$ and $x=2$.

\begin{equation}
    \begin{aligned}
        9=5+4=5 \cdot m+4 \\
        9^2=81=16 \cdot 5+1=5 \cdot m+1
    \end{aligned}
\end{equation}

Now for the subsequent cases we show that result does not change when we add 2 to the value of x.

\begin{equation}
    \begin{aligned}
        9^{x+2} \\
        9^x \cdot 9^2
    \end{aligned}
\end{equation} 

Now we have that $9^x$ x can either be odd or even if odd we assume that we get $5m+4$ and if even we assume $5m+1$ we know that $9^2=5m+1$ so let's see if the results change.

\begin{equation}
    \begin{aligned}
        (5m + 4) \cdot (5m + 1) \\ 
        (5m)^2 + 5m + 5 \cdot 4m + 4 \\
        5 \cdot (5m^2 + m + 4m) + 4
    \end{aligned}
\end{equation} 

\begin{equation}
    \begin{aligned}
        (5m + 1) \cdot (5m + 1) \\ 
        (5m)^2 + 5m + 5m + 1 \\
        5 \cdot (5m^2 + 2m) + 1
    \end{aligned}
\end{equation} 

This are the desired results thus Q(x) have been shown to be equivelant to R(x)

\subsection{Subtask 3}
\noindent
\textbf{Forklar om \eqref{eq:BetingelseReduceret} er på CNF, DNF, PCNF og PDNF.}

\bigskip
\noindent
\textbf{Solution}
