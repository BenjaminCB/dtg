Umiddelbart er vi ikke kommet nærmere hvilke $x$, der opfylder \eqref{eq:Betingelse}. Vi kan dog vise, at $Q(x)$ og $R(x)$ egentlig er betingelser, der minder meget om hinanden.


\section{Task 3}
\subsection{Subtask 1}
\noindent
\textbf{Bevis følgende sætning: Lad $x$ være et positivt heltal. At $\gcd(x,2)=1$ er ækvivalent med, at $x$ er ulige.}

\bigskip
\noindent
\textbf{Solution}

\subsection{Subtask 2}
\noindent
\textbf{Forklar hvorfor, at ovenstående betyder, at $Q(x)$ reelt set er det samme som $R(x)$. Derfor kan vi erstatte $R(x)$ med $Q(x)$ i \eqref{eq:Betingelse} og opnå, at \eqref{eq:Betingelse} er ækvivalent med
	\begin{align}\label{eq:BetingelseReduceret}
		&(P(x)\wedge \neg Q(x)) \vee \neg(P(x)\vee \neg Q(x) \vee Q(x)) \vee (\neg P(x) \wedge \neg Q(x) \wedge Q(x)) \nonumber\\ 
		\equiv& (P(x)\wedge \neg Q(x)) \vee \neg(P(x)\vee \mathbb{T}) \vee (\neg P(x) \wedge \mathbb{F}) \nonumber\\
		\equiv& P(x)\wedge \neg Q(x)
	\end{align}
}

\bigskip
\noindent
\textbf{Solution}

\subsection{Subtask 3}
\noindent
\textbf{Forklar om \eqref{eq:BetingelseReduceret} er på CNF, DNF, PCNF og PDNF.}

\bigskip
\noindent
\textbf{Solution}
